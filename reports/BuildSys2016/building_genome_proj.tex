% This is "sig-alternate.tex" V2.1 April 2013
% This file should be compiled with V2.5 of "sig-alternate.cls" May 2012
%
% This example file demonstrates the use of the 'sig-alternate.cls'
% V2.5 LaTeX2e document class file. It is for those submitting
% articles to ACM Conference Proceedings WHO DO NOT WISH TO
% STRICTLY ADHERE TO THE SIGS (PUBS-BOARD-ENDORSED) STYLE.
% The 'sig-alternate.cls' file will produce a similar-looking,
% albeit, 'tighter' paper resulting in, invariably, fewer pages.
%
% ----------------------------------------------------------------------------------------------------------------
% This .tex file (and associated .cls V2.5) produces:
%       1) The Permission Statement
%       2) The Conference (location) Info information
%       3) The Copyright Line with ACM data
%       4) NO page numbers
%
% as against the acm_proc_article-sp.cls file which
% DOES NOT produce 1) thru' 3) above.
%
% Using 'sig-alternate.cls' you have control, however, from within
% the source .tex file, over both the CopyrightYear
% (defaulted to 200X) and the ACM Copyright Data
% (defaulted to X-XXXXX-XX-X/XX/XX).
% e.g.
% \CopyrightYear{2007} will cause 2007 to appear in the copyright line.
% \crdata{0-12345-67-8/90/12} will cause 0-12345-67-8/90/12 to appear in the copyright line.
%
% ---------------------------------------------------------------------------------------------------------------
% This .tex source is an example which *does* use
% the .bib file (from which the .bbl file % is produced).
% REMEMBER HOWEVER: After having produced the .bbl file,
% and prior to final submission, you *NEED* to 'insert'
% your .bbl file into your source .tex file so as to provide
% ONE 'self-contained' source file.
%
% ================= IF YOU HAVE QUESTIONS =======================
% Questions regarding the SIGS styles, SIGS policies and
% procedures, Conferences etc. should be sent to
% Adrienne Griscti (griscti@acm.org)
%
% Technical questions _only_ to
% Gerald Murray (murray@hq.acm.org)
% ===============================================================
%
% For tracking purposes - this is V2.0 - May 2012

\documentclass{sig-alternate-05-2015}


\begin{document}

% Copyright
\setcopyright{acmcopyright}
%\setcopyright{acmlicensed}
%\setcopyright{rightsretained}
%\setcopyright{usgov}
%\setcopyright{usgovmixed}
%\setcopyright{cagov}
%\setcopyright{cagovmixed}


% DOI
\doi{XX}

% ISBN
\isbn{XX}

%Conference
\conferenceinfo{BuildSys '16}{November 16-17, 2016, Stanford, CA, USA}

\acmPrice{\$X}

%
% --- Author Metadata here ---
\conferenceinfo{BUILDSYS}{'16 Stanford, CA, USA}
%\CopyrightYear{2007} % Allows default copyright year (20XX) to be over-ridden - IF NEED BE.
%\crdata{0-12345-67-8/90/01}  % Allows default copyright data (0-89791-88-6/97/05) to be over-ridden - IF NEED BE.
% --- End of Author Metadata ---

\title{The Building Data Genome Project: A Collection of Public Datasets for Non-Residential Building Electrical Meter Characterization}
% \subtitle{[Extended Abstract]
% \titlenote{A full version of this paper is available as
% \textit{Author's Guide to Preparing ACM SIG Proceedings Using
% \LaTeX$2_\epsilon$\ and BibTeX} at
% \texttt{www.acm.org/eaddress.htm}}}
%
% You need the command \numberofauthors to handle the 'placement
% and alignment' of the authors beneath the title.
%
% For aesthetic reasons, we recommend 'three authors at a time'
% i.e. three 'name/affiliation blocks' be placed beneath the title.
%
% NOTE: You are NOT restricted in how many 'rows' of
% "name/affiliations" may appear. We just ask that you restrict
% the number of 'columns' to three.
%
% Because of the available 'opening page real-estate'
% we ask you to refrain from putting more than six authors
% (two rows with three columns) beneath the article title.
% More than six makes the first-page appear very cluttered indeed.
%
% Use the \alignauthor commands to handle the names
% and affiliations for an 'aesthetic maximum' of six authors.
% Add names, affiliations, addresses for
% the seventh etc. author(s) as the argument for the
% \additionalauthors command.
% These 'additional authors' will be output/set for you
% without further effort on your part as the last section in
% the body of your article BEFORE References or any Appendices.

\numberofauthors{4} %  in this sample file, there are a *total*
% of EIGHT authors. SIX appear on the 'first-page' (for formatting
% reasons) and the remaining two appear in the \additionalauthors section.
%
\author{
% You can go ahead and credit any number of authors here,
% e.g. one 'row of three' or two rows (consisting of one row of three
% and a second row of one, two or three).
%
% The command \alignauthor (no curly braces needed) should
% precede each author name, affiliation/snail-mail address and
% e-mail address. Additionally, tag each line of
% affiliation/address with \affaddr, and tag the
% e-mail address with \email.
%
% 1st. author
\alignauthor
Clayton Miller\\
       \affaddr{Institute of Technology in Architecture, ETH Z\"urich}\\
       \affaddr{Address}\\
       \affaddr{Z\"urich, Switzerland}\\
       \email{miller.clayton\\@arch.ethz.ch}
% 2nd. author
\alignauthor
Forrest Meggers\\
       \affaddr{Dept. of Architecture, Princeton University}\\
       \affaddr{Address}\\
       \affaddr{Princeton, NJ, USA}\\
       \email{fmeggers@princeton.edu}
% 3rd. author
\alignauthor Zolan Nagy\\
       \affaddr{Affiliation}\\
       \affaddr{Address}\\
       \affaddr{City, State}\\
       \email{nagy@arch.ethz.ch}
\and  % use '\and' if you need 'another row' of author names
% 4th. author
\alignauthor Arno Schlueter\\
       \affaddr{Institute of Technology in Architecture, ETH Z\"urich}\\
       \affaddr{Address}\\
       \affaddr{Z\"urich, Switzerland}\\
       \email{schlueter@arch.ethz.ch}
}
% There's nothing stopping you putting the seventh, eighth, etc.
% author on the opening page (as the 'third row') but we ask,
% for aesthetic reasons that you place these 'additional authors'
% in the \additional authors block, viz.
% \additionalauthors{Additional authors: John Smith (The Th{\o}rv{\"a}ld Group,
% email: {\texttt{jsmith@affiliation.org}}) and Julius P.~Kumquat
% (The Kumquat Consortium, email: {\texttt{jpkumquat@consortium.net}}).}
% \date{30 July 1999}
% Just remember to make sure that the TOTAL number of authors
% is the number that will appear on the first page PLUS the
% number that will appear in the \additionalauthors section.

\maketitle
\begin{abstract}
As of 2015, there are over 60 million smart meters installed in the United States. These data are at the forefront of \emph{big data} analytics for the building and construction industry. Despite the massive number of meters, only a few public data sources of hourly non-residential meter data exist for the purpose of testing clustering, classification or prediction algorithms. This paper describes the collection, cleaning, and compilation of several such data sets found publicly on-line, in addition to several collected by the authors. There are 507 whole building electrical meters in this collection and a majority are from buildings on University campuses. The intent of this collection is to serve as an initial repository of open, non-residential data sources that can be built upon by other researchers. Various characterization techniques such as temporal feature extraction, clustering, classification, and prediction are implemented on the data sets to illustrate the usefulness of such a repository. 
\end{abstract}


%
% The code below should be generated by the tool at
% http://dl.acm.org/ccs.cfm
% Please copy and paste the code instead of the example below. 
%
\begin{CCSXML}
<ccs2012>
 <concept>
  <concept_id>10010520.10010553.10010562</concept_id>
  <concept_desc>Computer systems organization~Embedded systems</concept_desc>
  <concept_significance>500</concept_significance>
 </concept>
 <concept>
  <concept_id>10010520.10010575.10010755</concept_id>
  <concept_desc>Computer systems organization~Redundancy</concept_desc>
  <concept_significance>300</concept_significance>
 </concept>
 <concept>
  <concept_id>10010520.10010553.10010554</concept_id>
  <concept_desc>Computer systems organization~Robotics</concept_desc>
  <concept_significance>100</concept_significance>
 </concept>
 <concept>
  <concept_id>10003033.10003083.10003095</concept_id>
  <concept_desc>Networks~Network reliability</concept_desc>
  <concept_significance>100</concept_significance>
 </concept>
</ccs2012>  
\end{CCSXML}

\ccsdesc[500]{Computer systems organization~Embedded systems}
\ccsdesc[300]{Computer systems organization~Redundancy}
\ccsdesc{Computer systems organization~Robotics}
\ccsdesc[100]{Networks~Network reliability}


%
% End generated code
%

%
%  Use this command to print the description
%
\printccsdesc

% We no longer use \terms command
%\terms{Theory}

\keywords{Open Data, Non-Residential Building Meter Data, Benchmark Data Set}

\section{Introduction}
% The \textit{proceedings} are the records of a conference.
% ACM seeks to give these conference by-products a uniform,
% high-quality appearance.  To do this, ACM has some rigid
% requirements for the format of the proceedings documents: there
% is a specified format (balanced  double columns), a specified
% set of fonts (Arial or Helvetica and Times Roman) in
% certain specified sizes (for instance, 9 point for body copy),
% a specified live area (18 $\times$ 23.5 cm [7" $\times$ 9.25"]) centered on
% the page, specified size of margins (1.9 cm [0.75"]) top, (2.54 cm [1"]) bottom
% and (1.9 cm [.75"]) left and right; specified column width
% (8.45 cm [3.33"]) and gutter size (.83 cm [.33"]).

% The good news is, with only a handful of manual
% settings\footnote{Two of these, the {\texttt{\char'134 numberofauthors}}
% and {\texttt{\char'134 alignauthor}} commands, you have
% already used; another, {\texttt{\char'134 balancecolumns}}, will
% be used in your very last run of \LaTeX\ to ensure
% balanced column heights on the last page.}, the \LaTeX\ document
% class file handles all of this for you.

% The remainder of this document is concerned with showing, in
% the context of an ``actual'' document, the \LaTeX\ commands
% specifically available for denoting the structure of a
% proceedings paper, rather than with giving rigorous descriptions
% or explanations of such commands.

\section{Data Sources}


\subsection{Data Cleaning and Standardization}


\subsection{Data Repository}


\section{Temporal Feature Extraction}

\subsection{}



\section{Classification}


\section{Prediction}

Prediction of electrical loads based on their shape and trends over time is a mature field developed to forecast consumption to detect anomalies and analyze the impact of demand response and efficiency measures. The most common technique in this category is the use of heating and cooling degree days to normalize monthly consumption \cite{fels_prism:_1986}. Over the years, various other techniques have been developed using techniques such as neural networks, ARIMA models, and more complex regression \cite{taylor_comparison_2006}. However, simplified techniques have retained their usefulness over time due to ease of implementation and accuracy. In the context of temporal feature creation, a regression model provides various metrics that describe how well a meter conforms to expected assumptions. For example, if actual measurements and predicted consumption match well, the underlying behavior of a energy-consuming systems in the building has been captured adequately. If not, there is uncharacterized phenomenon that will need to be captured with a different type of model or feature. 

A contemporary, simplified load prediction technique is selected to create temporal features that capture whether electrical measurement is simply a function of time-of-week scheduling. This model was developed by Matthieu et al. and Price and implemented mostly in the context of electrical demand response evaluation \cite{price_methods_2010, mathieu_quantifying_2011}. The premise of the model is based on two features: a time-of-week indicator and an outdoor air temperature dependence. This model is also known as the \emph{Time-of-week and Temperature or (TOWT)} model or \emph{LBNL regression model} and is implemented in the \emph{eetd-loadshape} library developed by Lawrence Berkeley National Laboratory\footnote{https://bitbucket.org/berkeleylab/eetd-loadshape}.

According to the literature, the model operates as follows \cite{price_methods_2010}. The time of week indicator is created by dividing each week into a set of intervals corresponding to each hour of the week. For example, the first interval is Sunday at 01:00, the second is Sunday at 02:00, and so on. The last, or 168th, interval is Saturday at 23:00. A different regression coefficient, $\alpha_i$, is calculated for each interval in addition to temperature dependence. The model uses outdoor air temperature dependence to divide the intervals into two categories: one for occupied hours and one for unoccupied. These modes are not necessarily indicators of exactly when people are inhabiting the building, but simply an empirical indication of when occupancy-related systems are detected to be operating. Seperate piecewise-continuous temperature dependencies are then calculated for each type of mode. The outdoor air temperature is divided into six equally-sized temperature intervals. A temperature parameter, $\beta_j$, with $j = 1...6$, is assigned to each interval. Within the model, the outdoor air temperature at time, $t$, occuring at time-of-week, $i$, (designated as $T(t_i)$) is divided into six component temperatures, $T_{c,j}(t_i)$. Each of these temperatures is multiplied by $\beta_j$ and then summed to determine the temperature-dependent load. For occupied periods the building load, $L_o$, is calculated by Equation \ref{eq:tempdepload}.

\begin{equation}
\label{eq:tempdepload}
L_0(t_i,T(t_i) = \alpha_i + \sum_{j=1}^{6}\beta_j T_{c,j}(t_i)
\end{equation}

Prediction of unoccupied mode occurs using a single temperature parameter, $\beta_u$. Unnoccupied load, $L_u$, is calculated with Equation \ref{eq:nontempdepload}.

\begin{equation}
\label{eq:nontempdepload}
L_0(t_i,T(t_i) = \alpha_i + \beta_u T_{c,j}(t_i)
\end{equation}

The primary means of temporal feature creation from this process is through the analysis of model fit. The first metric calculated is a normalized, hourly residual, $R$, that can be used to visualize deviations from the model. It is calculated from the actual load, $L_a$, and the predicted load, $L_p$. The residual at a specific hour, $t$, is calculated using Equation \ref{eq:residualerror}.

\begin{equation}
\label{eq:residualerror}
R_t = \frac{L_{t,a} - L_{t,p}}{max_{L_a}}
\end{equation}

An example of the TOWT model implemented on one of the case study buildings is seen in Figure \ref{fig:loadshape_single}. Two primary characteristics are captured from model residual analysis. The first is the building's primary deviation from a set time-of-week schedule and behavior causing the model to highly over-predict. These deviations are most often attributed to public holidays, breaks in normal operation, or changes in normal operating modes. In the single building study, one of the most obvious daily deviations, Christmas Day, is observed. This day is significantly over-predicted due to the model not being informed of the Christmas Day holiday. The automated capture of these phenomenon can inform whether the building is of a certain use-type or in a certain juristiction. The second characteristic captured are periods of under prediction when the building is consuming more electricity than expected. These data inform whether a building is being consistently utilized consistently, or whether there is volatility in its normal operating schedule from week-to-week. Figure \ref{fig:loadshape_all} illustrates an overview of implementation on all the buildings.




\section{Conclusions}
This paragraph will end the body of this sample document.
Remember that you might still have Acknowledgments or
Appendices; brief samples of these
follow.  There is still the Bibliography to deal with; and
we will make a disclaimer about that here: with the exception
of the reference to the \LaTeX\ book, the citations in
this paper are to articles which have nothing to
do with the present subject and are used as
examples only.
%\end{document}  % This is where a 'short' article might terminate

%ACKNOWLEDGMENTS are optional
\section{Acknowledgments}


%
% The following two commands are all you need in the
% initial runs of your .tex file to
% produce the bibliography for the citations in your paper.
\bibliographystyle{abbrv}
\bibliography{building_genome_proj}  % sigproc.bib is the name of the Bibliography in this case
% You must have a proper ".bib" file
%  and remember to run:
% latex bibtex latex latex
% to resolve all references
%
% ACM needs 'a single self-contained file'!
%
%APPENDICES are optional
%\balancecolumns
\appendix
%Appendix A
% \section{Headings in Appendices}
% The rules about hierarchical headings discussed above for
% the body of the article are different in the appendices.
% In the \textbf{appendix} environment, the command
% \textbf{section} is used to
% indicate the start of each Appendix, with alphabetic order
% designation (i.e. the first is A, the second B, etc.) and
% a title (if you include one).  So, if you need
% hierarchical structure
% \textit{within} an Appendix, start with \textbf{subsection} as the
% highest level. Here is an outline of the body of this
% document in Appendix-appropriate form:
% \subsection{Introduction}
% \subsection{The Body of the Paper}
% \subsubsection{Type Changes and  Special Characters}
% \subsubsection{Math Equations}
% \paragraph{Inline (In-text) Equations}
% \paragraph{Display Equations}
% \subsubsection{Citations}
% \subsubsection{Tables}
% \subsubsection{Figures}
% \subsubsection{Theorem-like Constructs}
% \subsubsection*{A Caveat for the \TeX\ Expert}
% \subsection{Conclusions}
% \subsection{Acknowledgments}
% \subsection{Additional Authors}
% This section is inserted by \LaTeX; you do not insert it.
% You just add the names and information in the
% \texttt{{\char'134}additionalauthors} command at the start
% of the document.
% \subsection{References}
% Generated by bibtex from your ~.bib file.  Run latex,
% then bibtex, then latex twice (to resolve references)
% to create the ~.bbl file.  Insert that ~.bbl file into
% the .tex source file and comment out
% the command \texttt{{\char'134}thebibliography}.
% % This next section command marks the start of
% % Appendix B, and does not continue the present hierarchy
% \section{More Help for the Hardy}
% The sig-alternate.cls file itself is chock-full of succinct
% and helpful comments.  If you consider yourself a moderately
% experienced to expert user of \LaTeX, you may find reading
% it useful but please remember not to change it.
%\balancecolumns % GM June 2007
% That's all folks!
\end{document}
